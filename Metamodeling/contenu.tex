\chapter*{Introduction}
	Blabla


\chapter{Meta-modèle}
	\section{Concept d'architecture en composants}
		
	
	
	\section{Éléments principaux}
		
	
	\section{Connexions}
		\subsection{Principes des roles, ports et services}
			
		\subsection{Bindings \& attachment}
			

\chapter{Modèle client/serveur}
		Le modèle qui décrit l'architecture de l'application Client/Serveur doit être divisé en deux niveaux de granularité afin de comprendre les interactions qui la régissent. En effet, le premier niveau décrit les relations entre les composants \textit{Client} et \textit{Serveur} reliés grâce à un connecteur tandis que le second détaille la structure du serveur : les composants qui séparent les différentes fonctionnalités et les éléments chargés de les faire communiquer. Nous allons décrire plus précisément tout ceci dans les sections suivantes.
		
	\section{Niveau 1 : vue générale}
		Le premier niveau décrit le fonctionnement général de l'application, en particulier les interactions entre le client et le serveur. 
		
		\subsection{Le client}
			Le composant client possède deux ports : un requis et un fourni
	\section{Niveau 2 : le serveur en détails}


\chapter*{Conclusion}
	
